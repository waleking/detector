% arara: pdflatex
% !arara: indent: {overwrite: on}
\documentclass{beamer}

\usetheme{CambridgeUS}

\usepackage{subfigure}

%-chage foot part-------------------------------
\makeatother
\setbeamertemplate{footline}
{
  \leavevmode%
  \hbox{%
  \begin{beamercolorbox}[wd=.17\paperwidth,ht=2.25ex,dp=1ex,center]{author in head/foot}%
    \usebeamerfont{author in head/foot}\insertshortauthor
  \end{beamercolorbox}%
  \begin{beamercolorbox}[wd=.46\paperwidth,ht=2.25ex,dp=1ex,center]{title in head/foot}%
    \usebeamerfont{title in head/foot}\insertshorttitle
  \end{beamercolorbox}%
  \begin{beamercolorbox}[wd=.37\paperwidth,ht=2.25ex,dp=1ex,center]{date in head/foot}% 
  \insertdate \hspace{0.2cm}
    \insertframenumber{} / \inserttotalframenumber\hspace*{1ex}
  \end{beamercolorbox}}%
  \vskip0pt%
}
\makeatletter
\setbeamertemplate{navigation symbols}{}
%--------------------------------

%-chage titlepage's font-------------------------------
\setbeamercolor{title}{bg=red!65!black,fg=white}
\usepackage{bm}
\usepackage{comment}
\usepackage{graphics} % used for scalebox
\usepackage{multirow}%used for multicolumn
\usepackage{array}%used for vrule
\usepackage{subfigure}
\usepackage{threeparttable} % used for footnotes in table

%---set tikz highlighter------------------------
\usepackage{multirow, booktabs, dcolumn, color} % Tables
\usepackage[beamer,customcolors]{hf-tikz}
\usetikzlibrary{calc}
% To set the hypothesis highlighting boxes red.
\tikzset{hl/.style={
    set fill color=red!80!black!40,
    set border color=red!80!black,
  },
}
%-----------------------------------------------

%---set tikz marker-----------------------------
\usepackage{tikz}
\usetikzlibrary{positioning}
\tikzset{>=stealth}

\newcommand{\tikzmark}[3][]{\tikz[overlay,remember picture,baseline] \node [anchor=base,#1](#2) {#3};}
%-----------------------------------------------

%---set textblock----------------------------
\usepackage[absolute,overlay]{textpos}
%--------------------------------------------------------

%use bold arrow
\usepackage{marvosym}

%used for aligning the figure to left
\usepackage[export]{adjustbox}

%used for colored block
\usepackage[listings,theorems]{tcolorbox}

%used for algorithm
\usepackage[ruled,vlined,linesnumbered]{algorithm2e}
\usepackage{algorithmic}

\setbeamertemplate{sidebar right}
{
  \vfill%
  \llap{\insertlogo\hskip0.1cm}%
  \vskip2pt%
  %\llap{\href{http://tex.stackexchange.com/}{A link to tex.sx}\hskip0.2cm}% NEW
  \vskip3pt% NEW
  \llap{\usebeamertemplate***{navigation symbols}\hskip0.1cm}%
  \vskip2pt%
}

\begin{document}

\title[Category-Level Transfer Learning from Knowledge Base to Microblog Stream for Accurate Event Detection]{Category-Level Transfer Learning from Knowledge Base to Microblog Stream for Accurate Event Detection}
\author[Weijing Huang, et,al.]{Weijing Huang, Tengjiao Wang, Wei Chen, Yazhou Wang}
\institute[EECS, Peking University]{School of Electronics Engineering and Computer Science, Peking University}
\date{\(\bm{@}\)DASFAA 2017, Suzhou, China}
\maketitle

%------------------------------
%page 2
\begin{frame}
\frametitle{Motivation}

Many Web applications need the \textbf{accurate event detection} technique on microblog stream, including:
\begin{enumerate}
	\item public opinion analysis [Chen, SIGIR 2013]
	\item public security [Li, ICDE 2012], [Imran, WWW 2014]
	\item disaster response [Sakaki,WWW 2010]
	\item breaking news report\footnote{\url{http://www.theverge.com/2016/12/1/13804542/reuters-algorithm-breaking-news-twitter}}
\end{enumerate}	
\vfill

But detecting events on twitter stream accurately is still challenging.
\end{frame}

%------------------------------
%page 3
\begin{frame}
\frametitle{Challenges (1/2)}
According to [Huang, WWW 2016], the challenges include,
\begin{enumerate}
\item fast changing
\item high noise
\item short length
\end{enumerate}	

\vfill

And, we found another key factor, 
\begin{enumerate}
\item Small events with fewer tweets \MVRightarrow \ \ Hard to trade off between precision and recall. 
\end{enumerate}




\end{frame}

%------------------------------
%page 4
\begin{frame}
\frametitle{Challenges (2/2)}	
Exploratory study on the \textit{Edinburgh twitter corpus}: 11/27 events contain less than 50 tweets.

\intextsep=5pt plus 3pt minus 1pt % 调整表格与正文的上间距
\begin{table}[]
\setlength{\abovecaptionskip}{0.cm}%set the distance between caption and table to 0 cm.
\setlength{\belowcaptionskip}{0.cm}
\centering
\caption{Statistics of labeled events.}
\label{my-label}
\scalebox{0.65}{
\begin{tabular}{|l|l|r|}
\hline
Event                                        & Date       & Event Size \\ \hline
S\&P downgrade US credit rating                  & 05/08/2011 & 656      \\ \hline
Atlantis shuttle lands                           & 21/07/2011 & 595      \\ \hline
US increases debt ceiling                        & 25/07/2011 & 485      \\ \hline
Plane with Russian hocky team Lokomotiv crashes  & 07/09/2011 & 286      \\ \hline
Amy Winehouse dies                               & 23/07/2011 & 283      \\ \hline
Gunman opens fire in youth camp in Norway        & 23/07/2011 & 260      \\ \hline
Earthquake in Virginia                           & 24/08/2011 & 246      \\ \hline
First victim of London riots dies                & 09/08/2011 & 174      \\ \hline
Explosion in French nuclear plant in Marcoule    & 12/09/2011 & 135      \\ \hline
Google announces plans to bury Motorola Mobility & 15/08/2011 & 127      \\ \hline
NASA announces there might be water on Mars      & 04/08/2011 & 124      \\ \hline
Car bomb explodes in Oslo, Norway                & 22/07/2011 & 114      \\ \hline
... & ... & ... \\ \hline
\tikzmarkin<1>[hl]{bH1}Indian and Bangladesh sign a border pact         & 06/09/2011 & 25       \\ \hline
Flight 4896 crash                                & 13/07/2011 & 21       \\ \hline
First aritficial organ transplant                & 12/07/2011 & 18       \\ \hline
three men die in riots in england                & 10/08/2011 & 16       \\ \hline
rebels capture interational tripoli airport      & 21/08/2011 & 13       \tikzmarkend{bH1} \\ \hline
\end{tabular}
}
\end{table}

 \begin{textblock}{0.1}(13.5,11.7)
  \footnotesize{11 Small events with fewer tweets}
 \end{textblock}

\end{frame}

%\begin{comment}
%------------------------------
%page 5
\begin{frame}
\frametitle{How about existing methods? (1/2)}
Event detection methods \textit{without extra information}, such as
\begin{enumerate}
\item clustering articles
\begin{itemize}
	\item LSH[Petrovic, NAACL 2010]
	\item need to set threshold to determine whether new article represents a new event.
\end{itemize}
\item analyzing word frequencies
\begin{itemize}
	\item EDCoW[Weng, ICWSM 2011]
	\item treat the word as the basic unit in analysis, without regarding polysemy words (words have different meanings, e.g. ``apple")
\end{itemize}
\item finding bursty topics via topic modeling
\begin{itemize}
	\item TimeUserLDA[Diao, ACL 2012], BurstyBTM[Yan, AAAI 2015]
	\item detects the ``large" events but may ignore the ``small" ones. 
\end{itemize}
\end{enumerate}
\end{frame}

\begin{frame}
\frametitle{How about existing methods? (2/2)}
Event detection methods \textit{by leveraging extra information}
\begin{enumerate}
	\item typical one: Twevent[Li, CIKM 2012]
	\begin{itemize}
		\item divides the tweet into segments according to the Microsoft Web N-Gram service and Wikpedia
		\item detects the bursty segments and cluster these segments into candidate events
		\item still has to trade off between precision and recall
		\begin{itemize}
			\item e.g., the bursty segment in the not-so-popular event ``\textit{first artificial organ transplant}" is missed
		\end{itemize}
	\end{itemize}
\end{enumerate}
\end{frame}


%------------------------------
%page 6
\begin{frame}
\textbf{Much easier} to detect the event on the time series of word ``hood" related to \textit{Military}, \textbf{without adjusting the threshold}.
\frametitle{An Example}	
\begin{figure}[h]
		\setlength{\abovecaptionskip}{0.cm}
        \setlength{\belowcaptionskip}{0.cm}
        \centering
        \includegraphics[width=1.0\columnwidth]{../img/hood.pdf}
        \caption{The comparison of the time series between the raw word \textit{hood} and the \textit{Military} related word \textit{hood}, computed on the \textit{Edinburgh twitter corpus}. Refer the event to \url{https://en.wikipedia.org/wiki/Fort_Hood\#2011_attack_plot}.}
        \label{fig:hood}
\end{figure}

\begin{figure}[h]
	\setlength{\abovecaptionskip}{0.cm}
	\setlength{\belowcaptionskip}{0.cm}
	\centering
        \subfigure[]{
                \includegraphics[height=2.12cm]{Hood(headgear)-Wikipedia.pdf}
        }
        \subfigure[]{
                \includegraphics[height=2.12cm]{Fort_Hood.pdf}
        }
\end{figure}


\end{frame}

%------------------------------
%page 7
\begin{frame}
\frametitle{The insights on the example}	
\begin{columns}[onlytextwidth, t]
\column{0.48\linewidth}
    \begin{tcolorbox}[colback=red!5,colframe=red!5]
    Knowledge Base
        \begin{itemize}
            \item Well organized
            \item Constructed elaborately
            \item full of rich information
        \end{itemize}
    \end{tcolorbox}

\column{0.48\linewidth}
    \begin{tcolorbox}[colback=red!5,colframe=red!5]
    Microblog stream
        \begin{itemize}
            \item Short length
            \item Fast changing
            \item High noise
        \end{itemize}
    \end{tcolorbox}
\end{columns}

\begin{tcolorbox}[colback=red!5,colframe=red!75!black]
The benefit of \textbf{enriching the semantics} and \textbf{filtering out noise} by Knowledge Base for microblogs is attractive.
\\~\\
But it's expensive to retrieve every word of tweets in the Knowledge Base.
\end{tcolorbox}


\end{frame}


%------------------------------
%page 8
\begin{frame}
\frametitle{Our solution}	
\textsc{TransDetector}: a novel category-level transfer learning method
\begin{itemize}
	\item transfer KB's \textbf{category-level info} into microblog stream
	\item balance the performance and the cost of leveraging knowledge base for event detection
\end{itemize}

\intextsep=5pt plus 3pt minus 1pt % 调整表格与正文的上间距
\begin{figure}[h]
		\setlength{\abovecaptionskip}{0.cm}
        \setlength{\belowcaptionskip}{0.cm}
        \centering
        \includegraphics[width=0.43\columnwidth]{../img/NSDetectorExample.pdf}
        \caption{\textsc{TransDetector}'s processing flow, in 3 phases.}
        \label{fig:hood}
\end{figure}


\end{frame}

\begin{frame}
\frametitle{\textsc{TransDetector}: Phase 1 (Extracting Category-Level Topics in KB) (1/3)}	
There is a three fold hierarchical structure in Knowledge Base.
\begin{enumerate}
	\item \textbf{Taxonomy Graph \(G^{(0)}\)}. The directed edges in \(G^{(0)}\) represent the \textit{class}\(\rightarrow\)\textit{subclass} relations in KB. 
		\begin{itemize}
			\item e.g., \textit{Main topic classifications} \(\rightarrow\) \textit{Society}
		\end{itemize}
	\item \textbf{Category-Page Bipartite Graph \(G^{(1)}\)}. The directed edges in \(G^{(1)}\) represent the \textit{class}\(\rightarrow\)\textit{instance} relations in KB.
		\begin{itemize}
			\item e.g., \textit{Military} \(\rightarrow\) \textit{page: Armed Forces}
		\end{itemize}
	\item \textbf{Page-Content Map \(G^{(2)}\)}. 
For a specific Wikipedia dumps version, the edges  \textit{page} \(\rightarrow\)\textit{content} in \(G^{(2)}\) define a one-to-one mapping.
		\begin{itemize}
			\item e.g., \textit{page: Armed Forces} \(\rightarrow\) \textit{content(20170325version): ``The armed forces of a country are its government-sponsored defense, fighting forces, and organizations..."}
		\end{itemize}
\end{enumerate}
\end{frame}

\begin{frame}
\frametitle{\textsc{TransDetector}: Phase 1 (Extracting Category-Level Topics in KB) (2/3)}
\begin{algorithm}[H]
\scriptsize
\caption{Extraction of Category-Level Topics in Knowledge Base}
\label{alg:normalStatesInit}

\KwIn{Taxonomy's Graph \(G^{(0)}\), Category-Page Bipartite Graph \(G^{(1)}\), Page-Content Bipartite Graph \(G^{(2)}\), topic related category node \(c\)}
\KwOut{\(c\)'s category-level topic in knowledge base \(\bm{h}_c\)}
\(Pages(c)\leftarrow \varnothing\), \(\bm{h}_c \leftarrow \varnothing\)\\
DAG \(G^{(0)'} \leftarrow\) Remove Cycles of \(G^{(0)}\) by nodes' HITS-PageRank scores. \label{alg:line2inNormalStatesInit}\\
\(SuccessorNodes(c) \leftarrow \) Breadth-first-traverse(\(G^{(0)'},c\))\label{alg:line3inNormalStatesInit}\\
\For{\(node \in SuccessorNodes(c)\)}{
    \(Pages(c) \leftarrow Pages(c) \cup G^{(1)}.neighbours(node)\) \label{alg:line5inNormalStatesInit}\\
}
Word frequency table \(n(c,.) \leftarrow \) do word count on the text contents of \(Pages(c)\) \\
Word frequency table \(n(All,.) \leftarrow \) do word count on the text contents of all pages in \(G^{(2)}\).\\
\For{word \(w\) in WordFrequencyTable(All).keys()}{
    \(chi(c,w) \leftarrow \) \(w\)'s chi-square statistics on \(WordFrequencyTable(c)\) and \(WordFrequencyTable(All)\).\\
    \(h_{c,w} \leftarrow chi(c,w)\) \label{alg:line10inNormalStatesInit}\\
}
\Return{\(\bm{h}_c\)}
\end{algorithm}
	
\end{frame}


%\begin{comment}
%------------------------------
%page 9
\begin{frame}
\frametitle{\textsc{TransDetector}: Phase 1 (Extracting Category-Level Topics in KB) (3/3)}	
Taking the category \textit{Military} as an example, we extract \textit{Military}'s category-level topic \(\bm{h_{Military}}\).
\begin{figure}[h]
		\setlength{\abovecaptionskip}{0.cm}
        \setlength{\belowcaptionskip}{0.cm}
        \centering
        \includegraphics[width=0.45\columnwidth]{../img/initializationExample.pdf}
        \caption{Extracting Category-Level Topics in Knowledge Base via its three fold hierarchical structure, taking \textit{Military} as an example.}
        \label{fig:hood}
\end{figure}

\end{frame}

%------------------------------
%page 10
\begin{frame}
\frametitle{\textsc{TransDetector}: Phase 2 (Transferring Category-Level Info into Microblog Stream) (1/2)}	
Transfer KB's Category-Level Topics \(\{\bm{h_c}\}_{c=1}^{K_{KB}}\) into microblogs stream: CTrans-LDA.
\begin{figure}[h]
		\setlength{\abovecaptionskip}{0.cm}
        \setlength{\belowcaptionskip}{0.cm}
        \centering
        \includegraphics[width=0.6\columnwidth]{../model/lda_tikz.pdf}
        \caption{Diagram of CTrans-LDA.}
        \label{fig:hood}
\end{figure}

In CTrans-LDA, \(\{\bm{h_c}\}_{c=1}^{K_{KB}}\) is used as prior information:
\setlength{\abovedisplayskip}{0pt}
\setlength{\belowdisplayskip}{0pt}
\begin{scriptsize}
\begin{equation}
\label{eq:wikiPrior}
\begin{aligned}
\tau_{kv}=
\left\{ \begin{aligned}
\lambda \frac{h_{kv}}{\sum_{v\in S_{k}}h_{kv}} &,v\in S_{k}\ and  \ k \leq K_{\bm{KB}} \\
0&,v \notin S_{k} \ or \ k > K_{\bm{KB}} \\
\end{aligned}\right.
\end{aligned}
\end{equation}
\end{scriptsize}
\end{frame}

\begin{frame}
\frametitle{\textsc{TransDetector}: Phase 2 (Transferring Category-Level Info into Microblog Stream) (2/2)}	
We use Gibbs Sampling for solving CTrans-LDA.
\begin{itemize}
	\item The initialization probability \(\hat{q}_{k|v}\) makes sure that the learned topics are aligned to the pre-defined category-level topic.
\setlength{\abovedisplayskip}{0pt}
\setlength{\belowdisplayskip}{0pt}
\begin{scriptsize} 
\begin{equation}
\label{eq:initProbability}
\begin{aligned}
\hat{q}_{k|v}=
\left\{ \begin{aligned}
\frac{\tau_{kv}}{\sum_{k=1}^{K}\tau_{kv}} &,\sum_{k}\tau_{kv}>0 & (a)\\
0&, \sum_{k}\tau_{kv}=0 \ and \ k \leq K_{\bm{KB}} & (b)\\
1/(K-K_{\bm{KB}})&,\sum_{k}\tau_{kv}=0 \ and \ k > K_{\bm{KB}} & (c)
\end{aligned}\right.
\end{aligned}
\end{equation}
\end{scriptsize}
\item Conditional probability in gibbs sampling:
\(p(z_{dn}=k|.)\propto (n^{(d)}_{dk}+\alpha m_k)(n^{(w)}_{kv}+\tau_{kv}+\beta)/(n^{(w)}_{k,.}+\tau_{k,.}+V\beta)\).
\end{itemize}

\end{frame}


%------------------------------
%page 11
\begin{frame}
\frametitle{\textsc{TransDetector}: Phase 3 (Detecting Events on Category-Level Word Time Series) (1/2)}	
After transfer learning, we conduct analysis on category-level word time series, and detect events in microblog stream.
\begin{figure}[h]
		\setlength{\abovecaptionskip}{0.cm}
        \setlength{\belowcaptionskip}{0.cm}
        \centering
        \includegraphics[width=.99\columnwidth]{../img/screenShot.png}
        \caption{Visualizing Category-Level Word Time Series in Microblog Stream on Edinburgh Twitter Corpus (20110711-20110915), taking \textit{Military} as an example.}
        \label{fig:hood}
\end{figure}
\end{frame}

\begin{frame}
\frametitle{\textsc{TransDetector}: Phase 3 (Detecting Events on Category-Level Word Time Series) (2/2)}
With richer semantics, and fewer noise, we detect events more accurately, in all the following granularities.(see more technique details in our paper)
\begin{enumerate}
	\item events' candidate words
	\begin{itemize}
		\item e.g., \textit{Ft.}, \textit{Hood}, \textit{attack}.
	\end{itemize}
	\item event phrases
	\begin{itemize}
		\item e.g., \textit{Ft. Hood attack}.
	\end{itemize}
	\item events' representative microblogs
	\begin{itemize}
		\item e.g., \textit{Possible Ft. Hood Attack Thwarted \url{http://t.co/BSJ33hk}}.
	\end{itemize}
\end{enumerate}
\end{frame}

\begin{frame}
\frametitle{Experiment Settings}
\textbf{Dataset}
\begin{itemize}
	\item Knowledge Base. Wikipedia dumps\footnote{\url{https://dumps.wikimedia.org/enwiki/latest/enwiki-latest-categorylinks.sql.gz}}\footnote{\url{https://dumps.wikimedia.org/enwiki/latest/enwiki-latest-pages-articles.xml.bz2}}
	\item Microblog Stream. Edinburgh twitter corpus\footnote{\url{http://demeter.inf.ed.ac.uk/cross/docs/fsd_corpus.tar.gz}}
\end{itemize} 

\textbf{Baseline Methods}
\begin{itemize}
	\item Twevent, BurstyBTM, LSH, EDCoW, and TimeUserLDA
\end{itemize}
\textbf{Ground Truth}
\begin{itemize}
	\item Benchmark1 is labeled events in previous study.
	\item Benchmark2 is our manually checked events based on Twevent, BurstyBTM, LSH, EDCoW, TimeUserLDA and \textsc{TransDetector}.
\end{itemize}
\end{frame}


%------------------------------
%page 12
\begin{frame}
\frametitle{Experimental Results}
Evaluation on Categroy-Level Topics in Knowledge Base. On \textit{Aviation} topic, semantic coherence is much better than LightLDA (same as LDA) in terms of NPMI[Röder, WSDM 2015]\footnote{\url{https://github.com/AKSW/Palmetto}}. 

\begin{table}[h]
\setlength{\abovecaptionskip}{0.cm}%set the distance between caption and table to 0 cm.
\setlength{\belowcaptionskip}{0.cm}
\centering
\caption{The comparison on the topic coherence(NPMI) between our method and LightLDA, taking \textit{Aviation} as an example. (NPMI is computed on a group of ten words. \(\sim\) stands for the top five words.)}
\scalebox{0.45}{
\begin{tabular}{|c|l|l|c !{\vrule width 1pt} c|l|l|c|}
\hline
\multicolumn{4}{|c!{\vrule width 1pt}}{Category-Level Topics extracted from Wikipedia by \textsc{TransDetector}} & \multicolumn{4}{c|}{Topics Learned from Wikipedia by LightLDA}    \\ \hline
GID& \#words*  & words  & NPMI & GID& \#words*& words & NPMI \\ \hline
- & 1-5 & aircraft air airport flight airline &-& - & 1-5 & engine aircraft car air power &-\\ \hline
0 & 1-5, 6-10 & \(\sim\), airlines aviation flying pilot squadron &  0.113 & 0 & 1-5, 6-10 & \(\sim\), design flight model production speed & 0.112\\ \hline
1 & 1-5, 11-15 & \(\sim\), flights pilots raf airways fighter & 0.155 & 1 & 1-5, 11-15 &\(\sim\), system vehicle cars engines mm & 0.062\\ \hline
2 & 1-5, 16-20 & \(\sim\), boeing runway force crashed flew   & 0.092 & 2 & 1-5, 16-20 & \(\sim\), fuel vehicles designed models type & 0.072\\ \hline
3 & 1-5, 21-25 &\(\sim\), airfield landing passengers plane aerial & 0.179 & 3 & 1-5, 21-25 & \(\sim\), version front produced rear electric & 0.035\\ \hline
4 & 1-5, 26-30 &\(\sim\), bomber radar wing bombers crash & 0.137 & 4 & 1-5, 26-30 & \(\sim\), space control motor standard development & 0.085\\ \hline
5 & 1-5, 31-35 &\(\sim\), airbus airports operations jet helicopter & 0.189 & 5 & 1-5, 31-35 & \(\sim\), film range light using available & -0.002\\ \hline
6 & 1-5, 36-40 &\(\sim\), squadrons base flown havilland crew & 0.088 & 6 & 1-5, 36-40 & \(\sim\), wing powered wheel weight launch & 0.087\\ \hline
7 & 1-5, 41-45 & \(\sim\), combat luftwaffe aerodrome carrier fokker & 0.159 & 7 & 1-5, 41-45 & \(\sim\), developed low test ford cylinder & 0.007\\ \hline
8 & 1-5, 46-50 &\(\sim\), planes fly engine takeoff fleet & 0.186 & 8 & 1-5, 46-50 & \(\sim\), equipment side pilot hp aviation & 0.091\\ \hline
9 & 1-5, 51-55 &\(\sim\), fuselage helicopters aviator naval aero & 0.157 & 9 & 1-5, 51-55 & \(\sim\), systems us sold body drive & -0.051\\ \hline
10 & 1-5, 56-60 &\(\sim\), glider command training balloon faa & 0.166 & 10 & 1-5, 56-60 & \(\sim\), gear introduced class safety seat & 0.069\\ \hline
\(\cdots\) & \(\cdots\) &\(\cdots\) &\(\cdots\) & \(\cdots\) & \(\cdots\) &\(\cdots\) &\(\cdots\)\\ \hline
18 & 1-5, 96-100 &\(\sim\), scheduled carriers military curtiss biplane &0.131 & 18 & 1-5, 96-100 & \(\sim\), transmission special replaced limited different & 0.059\\ \hline
19 & 1-5, 101-105 &\(\sim\), accident engines iaf albatross rcaf &0.068 & 19 & 1-5, 101-105 & \(\sim\), features machine nuclear even unit & 0.011\\ \hline
\end{tabular}
}
\label{tbl:NPMIDetails}
\end{table}
	
\end{frame}


\begin{frame}
\frametitle{Experimental Results}	
Evaluation on Categroy-Level Topics in Knowledge Base.
On more topics, semantic coherence is much better than LightLDA (same as LDA) in terms of NPMI[Roder, WSDM 2015]. 
\begin{figure}[h]
	\setlength{\abovecaptionskip}{0.cm}
	\setlength{\belowcaptionskip}{0.cm}
        \centering
        \includegraphics[width=1.0\columnwidth]{../img/NPMI.pdf}
        \caption{More topics are compared at the NPMI metrics between our method and LightLDA}
        \label{fig:NPMI}
\end{figure}
\end{frame}





\begin{frame}
\frametitle{Experimental Results}	
Effectiveness of transferring category-level topics into the microblog stream, and finding more new words in the stream which is not stored in the knowledge base.

\begin{table}
\setlength{\abovecaptionskip}{0.cm}%set the distance between caption and table to 0 cm.
\setlength{\belowcaptionskip}{0.cm}
\centering
\caption{Category-Level Topics extracted from knowledge base and the corresponding topics on microblog stream learned from CTrans-LDA. The words in \textbf{\textit{bold}} font are newly learned on the microblog stream by the transfer learning.}
\scalebox{0.55}{
\begin{tabular}{|cc|cc|cc|cc|cc|cc|}
\hline
\multicolumn{2}{|c|}{\textit{Aviation}} & \multicolumn{2}{c|}{\textit{Health}} & \multicolumn{2}{c|}{\textit{Middle East}} & \multicolumn{2}{c|}{\textit{Military}} & \multicolumn{2}{c|}{\textit{Mobile Phones}}\\
\begin{tabular}[c]{@{}c@{}}Knowledge\\ Base\end{tabular} & \begin{tabular}[c]{@{}c@{}}Microblog\\ Stream\end{tabular} & \begin{tabular}[c]{@{}c@{}}Knowledge\\ Base\end{tabular} & \begin{tabular}[c]{@{}c@{}}Microblog\\ Stream\end{tabular} & \begin{tabular}[c]{@{}c@{}}Knowledge\\ Base\end{tabular} & \begin{tabular}[c]{@{}c@{}}Microblog\\ Stream\end{tabular} & \begin{tabular}[c]{@{}c@{}}Knowledge\\ Base\end{tabular} & \begin{tabular}[c]{@{}c@{}}Microblog\\ Stream\end{tabular} & \begin{tabular}[c]{@{}c@{}}Knowledge\\ Base\end{tabular} & \begin{tabular}[c]{@{}c@{}}Microblog\\ Stream\end{tabular} \\ 
\hline
aircraft & air & health & weight & al & \textbf{\textit{\#syria}} & army & killed & android & iphone\\ 
air & plane & patients & loss & israel & \textbf{\textit{\#bahrain}} & military & news & mobile & apple \\ 
airport & flight & medical & diet & iran & people & air & \textbf{\textit{\#libya}} & nokia & android \\ 
flight & time & disease & health & arab & israel & command & libya & ios & app \\
airline & airlines & treatment & cancer & israeli & police & force & rebels & phone & ipad \\
airlines & news & hospital & lose & egypt & \textbf{\textit{\#libya}} & regiment & people & samsung & samsung \\
aviation & boat & patient & fat & egyptian & \#egypt & forces & police & game & mobile\\
flying & airport & clinical & tips & ibn & news & squadron & war & app & blackberry \\
pilot & force & symptoms & treatment & jerusalem & \textbf{\textit{\#israel}} & infantry & libyan & iphone & tablet \\
squadron & fly & cancer & body & syria & world & battle & attack & htc & apps\\
\hline
\end{tabular}
}
\label{tbl:historyStates}
\end{table}
\end{frame}

\begin{frame}
\frametitle{Experimental Results}	
Effectiveness of event detection:
\begin{enumerate}
	\item \textsc{TransDetector} performs better in terms of the precision and the recall.
	\item only sacrificing in the DERate slightly because an event could be grouped into multiple categories.
	\begin{itemize}
		\item the event \textit{``S\&P downgrade US credit rating"}, related to \textit{politics} and  \textit{financial} simultaneously.
	\end{itemize}
\end{enumerate}

\begin{table}[h]
\setlength{\abovecaptionskip}{0.cm}%set the distance between caption and table to 0 cm.
\setlength{\belowcaptionskip}{0.cm}
\centering
\caption{Overall Performance on Event Detection}

\scriptsize
\scalebox{0.75}{
\begin{threeparttable}  

\begin{tabular}{|c|c|c|c|c|c|c|}
    \hline
    Method & \begin{tabular}[c]{@{}c@{}}Number of\\Events to \\ be Evaluated \end{tabular} & \begin{tabular}[c]{@{}c@{}}Recall@ \\ Benchmark1\end{tabular}& \begin{tabular}[c]{@{}c@{}}Precision@ \\ Benchmark2\end{tabular} & \begin{tabular}[c]{@{}c@{}}Recall@ \\ Benchmark2\end{tabular} & \begin{tabular}[c]{@{}c@{}}F@ \\ Benchmark2\end{tabular} & \begin{tabular}[c]{@{}c@{}}DERate\tnote{a}\ \ (Duplicate\\ Event Rate)@\\ Benchmark2\end{tabular} \\ \hline
    LSH & 500 & 0.704 & 0.788 & 0.651 & 0.713 & 0.348 \\ \hline
    TimeUserLDA & 100 & 0.370 & 0.790 & 0.177 & 0.289 & 0.114 \\ \hline
    Twevent & 375 &  0.741 & 0.808 & 0.658 & 0.725 & 0.142 \\ \hline
    EDCoW & 349 & 0.556 & 0.748 & 0.511 & 0.607 & 0.226 \\ \hline
    BurstyBTM & 200 & 0.667 & 0.825 & 0.384 & 0.497 & \textbf{0.079} \\ \hline
    \textsc{TransDetector} & 457 & \textbf{0.889} & \textbf{0.912} & \textbf{0.876} & \textbf{0.894} & 0.170 \\ \hline
    \end{tabular}

\begin{tablenotes}  
\item[a] DERate = (the number of duplicate events) / (the total number of detected realistic events)
\end{tablenotes}  
\end{threeparttable}  
}
\label{tbl:overall}
\end{table}
\end{frame}

\begin{frame}
\frametitle{Experimental Results}
To understand why \textsc{TransDetector} performs better.
\vfill
Show the relationship between the recall and the event size.
\begin{figure}[h]
	\setlength{\abovecaptionskip}{0.cm}
	\setlength{\belowcaptionskip}{0.cm}
        \centering
        \includegraphics[width=1.0\columnwidth]{../img/barchartOnBenchmark1.pdf}
        \caption{The relation between the recall and the event size}
        \label{fig:Benchmark1}
\end{figure}
\end{frame}


\begin{frame}
\frametitle{Experimental Results}	
To understand why \textsc{TransDetector} performs better.\\
Show the relationship between the recall and the event size, taking the \textit{military}-related events as an example.
\begin{table}
\setlength{\abovecaptionskip}{0.cm}%set the distance between caption and table to 0 cm.
\setlength{\belowcaptionskip}{0.cm}
\centering
\caption{Events about \textit{military} detected by systems between 2011-07-22 and 2011-07-28}
\label{my-label}
\scalebox{0.6}{
\begin{threeparttable}  
\begin{tabular}{|c|l|l|c|c|c|c|c|c|c|}
\hline
\multirow{2}{*}{Date} & \multirow{2}{*}{Event key words} & \multirow{2}{*}{Representative event tweet} & \multirow{2}{*}{\begin{tabular}[c]{@{}l@{}}Number of \\ event tweet\end{tabular}} & \multicolumn{6}{c|}{Methods\tnote{a}} \\ \cline{5-10} 
 &  &  &  & L & TU & TW & E & B & TD \\ \hline
7/22/11 & \begin{tabular}[c]{@{}l@{}}Norway, Oslo,\\ attacks, bombing\end{tabular} & \begin{tabular}[c]{@{}l@{}}Terror Attacks Devastate Norway: A bomb\\ ripped through government offices in Oslo \\and a gunman... http://dlvr.it/cLbk8\end{tabular} & 557 & \checkmark & \checkmark & \checkmark & \checkmark & \checkmark & \checkmark \\ \hline
7/23/11 & Gunman, rink & \begin{tabular}[c]{@{}l@{}}Gunman Kills Self, 5 Others at Texas Roller\\ Rink http://dlvr.it/cLcTH\end{tabular} & 43 & - & - & \checkmark &  \checkmark & - & \checkmark \\ \hline
7/26/11 & \begin{tabular}[c]{@{}l@{}}Kandahar, mayor, \\ suicide, attack\end{tabular} & \begin{tabular}[c]{@{}l@{}}TELEGRAPH{]}: Kandahar mayor killed by\\ Afghan suicide bomber: The mayor of \\Kandahar, the biggest city in south \_\end{tabular} & 47 & \checkmark & - & \checkmark & \checkmark & - & \checkmark \\ \hline
7/28/11 & Ft., Hood, attack & \begin{tabular}[c]{@{}l@{}} Possible Ft. Hood Attack Thwarted\\ http://t.co/BSJ33hk\end{tabular} & 52 & - & - & - & - & - & \checkmark \\ \hline
7/28/11 & \begin{tabular}[c]{@{}l@{}}Libyan, rebel, \\ gunned\end{tabular} & \begin{tabular}[c]{@{}l@{}}Libyan rebel chief gunned down in Benghazi \\ http://sns.mx/prfvy1\end{tabular} & 44 & - & - & - & - & - & \checkmark \\ \hline
\end{tabular}

\begin{tablenotes}  
\item[a] L=LSH, TU=TimeUserLDA, TW=Twevent, E=EDCoW, B=BurstyBTM, TD=\textsc{TransDetector}.
\end{tablenotes}  
\end{threeparttable}  
}
\end{table}

\end{frame}

\begin{frame}
\frametitle{Conclusions}	
\begin{itemize}
	\item Knowledge base is constructed elaborately and contains rich information, which can benefit the not-well-organized microblog stream. 
	\item We propose \textsc{TransDetector} method, and
	\begin{itemize}
		\item use category-level topic in knowledge base as the prior knowledge,
		\item transfer abundant knowledge from knowledge base into microblog stream
		\item enrich the semantics of microblogs and further enhances the accuracy of microblogs event detection
	\end{itemize}
	
\end{itemize}
\end{frame}



\begin{frame}
%\frametitle{A first slide}

\begin{center}
\Huge Thanks!
\end{center}

\begin{center}
\Huge Q\&A
\end{center}

%\begin{figure}[h]
%	\setlength{\abovecaptionskip}{0.cm}
%	\setlength{\belowcaptionskip}{0.cm}
%        \centering
%        \includegraphics[width=0.3\columnwidth]{code.png}
%\end{figure}


\footnotetext{This slide and more data are available at \url{http://q-r.to/bajx8I}}

\end{frame}

%\end{comment}


\end{document}
